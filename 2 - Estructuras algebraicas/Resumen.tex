% Created 2020-01-01 Wed 18:39
% Intended LaTeX compiler: pdflatex
\documentclass[a4paper, 11pt]{extarticle}
             \usepackage[utf8]{inputenc}
\usepackage[left=2cm, right=2cm, bottom=2.5cm, top=2.5cm]{geometry}
% Paquetes de matemáticas
\usepackage{amsmath, amsfonts, amssymb, commath}
\usepackage{tikz}
\usepackage{tikz-cd}
\newcommand{\tikzcircle}[2][red,fill=red]{\tikz[baseline=-0.5ex]\draw[#1,radius=#2] (0,0) circle ;}%
% Ajustes de idioma, gráficos, etc
\usepackage{adjustbox}
\usepackage{float}
\usepackage{hyperref}
\usepackage{graphicx}
\usepackage{gensymb}
\usepackage[spanish, english]{babel}
\usepackage{tikz}
\usepackage{multicol}
\usepackage{listings}
\usepackage{enumitem}
\setlist{nolistsep}
\usepackage{booktabs}
\usepackage{xcolor}
\usepackage{wrapfig}
%Fuentes.
% Alegreya tiene este toque antiguo con serifa, y la tipografía de las ecuaciones es también interesante.
% Gillius no tiene serifa, y también es equilibrada.
\usepackage[T1]{fontenc}
%\usepackage[default]{gillius}
\usepackage{newpxtext, newpxmath}
% Paquete para añadir Creative Commons al final del documento
\usepackage[
type={CC},
modifier={by-nc-nd},
version={3.0},
]{doclicense}
% Propiedades de párrafo
\setlength{\parindent}{0em}
\setlength{\parskip}{1.1em}
\renewcommand{\baselinestretch}{1.05}
\setlength\itemsep{0em}
% Definición de comandos. Muchos de ellos han surgido para Geometría, aunque se
% irá actualizando la lista. POSIBLEMENTE LO INTRODUZCA COMO COMANDOS DE ORG
\newcommand{\m}{\text{medio}}
\newcommand{\iso}{\text{Isom}}
% Para incluir mathcal en las ecuaciones. El /mathcal para Alegreya es el viejo
% y floritural estilo que odio.
\usepackage{calrsfs}
\DeclareMathAlphabet{\pazocal}{OMS}{zplm}{m}{n}
% Definición de colores agradables a la vista
\definecolor{azul}{HTML}{107896}
\definecolor{naranja}{HTML}{C2571A}
\definecolor{rojo}{HTML}{9A2617}
\definecolor{amarillo}{HTML}{BCA136}
\definecolor{verde}{HTML}{829356}
\definecolor{gris}{HTML}{909090}
\definecolor{rosa}{HTML}{F9A7B0}
\definecolor{amarillochillon}{HTML}{FBB117}
% Definición de comandos para teoremas, etc. El comando también incluye
% como argumento un texto, del estilo Teorema 3.5
\newcommand{\axioma}[1]{\textcolor{naranja}{\textbf{Axioma #1}}}
\newcommand{\tma}[1]{\textcolor{rojo}{\textbf{Teorema #1}}}
\newcommand{\propo}[1]{\textcolor{rojo}{\textbf{Proposición #1}}}
\newcommand{\defi}[1]{\textcolor{azul}{\textbf{Definición #1}}}
\newcommand{\obs}[1]{\textcolor{verde}{\textbf{Observación #1}}}
\newcommand{\ejem}[1]{\textcolor{verde}{\textbf{Ejemplo #1}}}
\newcommand{\ej}[1]{\textcolor{amarillo}{\textbf{Ejercicio #1}}}
\newcommand{\lema}[1]{\textcolor{rosa}{\textbf{Lema #1}}}
\newcommand{\cor}[1]{\textcolor{rosa}{\textbf{Corolario #1}}}
% La demostración es igual pero va con una letra más pequeña y en gris.
\newcommand{\dem}[1]{\textcolor{gris}{\small{Demostración. #1}}}
% Esto pone un triangulito de peligro para cuando algo es importante.
\newcommand{\importante}{\tikzcircle[amarillo, fill=amarillo]{4pt}\,}
% Para usar columnas emplea este trozo de código
% \begin{multicols*}{2}
% [\section{Axiomas para la geometría euclidiana plana}]
% 	\axioma{P1} Si tenemos el conjunto $\P$, denominado \textbf{plano}, y la aplicación $d:\P \times \P \rightarrow \R$ llamada \textbf{distancia}, entonces$(\P, d)$ es un espacio métrico.

\defi{2.2} Una \textbf{recta} $r \subset \P$ satisface
\begin{itemizex}
	\item $r$ contiene al menos dos puntos.
	\item Para toda terna de puntos $A, B, C$, están alineados si están en $r$.
\end{itemizex}

\axioma{P2} $\P$ contiene al menos tres puntos no alineados; y por dos puntos distintos, $A$ y $B$ de $\P$ pasa una recta, $r_{AB}$.

\defi{2.6} / \tma{2.7} Dos rectas se cortan si sólo tienen un punto en común, y si no tienen ningún punto en común, entonces se denominan \textbf{paralelas}, y se denota por $a \parallel b$. Dos rectas, o se cortan o son paralelas.

\importante\axioma{P3} Para toda recta $r \subset \P$ existe una biyección $\gamma: r \rightarrow \R$ tal que $|\gamma(X) - \gamma(Y)| = |x - y| = d(X, Y) \;\; \forall \;\; X,Y \in r$ 

\obs{2.8} Si $A, B \in r$ son distintos, entonces existe un punto $M\in r: d(A,M) = d(M,B)$ que denotamos por $\m[A,B]$ y se llama \textbf{punto medio}. Asimismo sólo existe un punto $B \in r$ tal que $B = \m[A, M]$.

\obligatorio \dem{Tomamos una biyección $\gamma: r \rightarrow \R$ y suponemos $\gamma(A) = a, \gamma(B) = b$. Para $X \in r$ tomamos $\gamma(X) = t$. Si suponemos $A \neq B$ entonces $d(A,X) = d(X,B) \iff \abs{(t-a)} = \abs{(t-b)} \iff t = \frac{a+b}{2}$. Por tanto, $M$ sólo puede ser $\gamma^{-1}(t)$. Si definimos, por tanto $M = \gamma^{-1}(\frac{a+b}{2})$ se tiene que 
$$d(A,M) = \abs{\gamma(M)-\gamma(A)} = \abs{\frac{a+b}{2}-a} = \frac{1}{2}\abs{b-a} = \frac{1}{2}d(A,B)$$}

\obs{2.9} Si $r$ es una recta y $P \in r$, entonces $r$ se puede dividir en dos \textbf{semirrectas}, que son los conjuntos $\{X \in r \; | \; \gamma(X) > \gamma(P)\}$ y $\{X \in r \; | \; \gamma(X) < \gamma(P)\}$.

\axioma{P4} Para toda recta $r \subset \P$ hay dos subconjuntos $H^1$ y $H^2$, denominados \textbf{semiplanos} de $r$, que verifican:
\begin{itemizex}
	\item $H^1 \cup H^2 = \P-r$
	\item Si $X,Y \in H^i$ entonces $[X,Y] \subset H^i$
	\item Si $X \in H^1$ y $Y \in H^2$ entonces $[X,Y] \cap r \neq \emptyset$.
\end{itemizex}

\defi{2.15} Sean $P, Q, R$ no alineados, entonces el triángulo $\triangle\{P,Q,R\}$, o $\triangle PQR$ está formado por los segmentos $[P,Q]$, $[Q,R]$, $[P,R]$, llamados lados, y los vértices $P,Q, R$.

\tma{2.16 [Axioma de Pasch]a} Dado un triángulo $\triangle PQR$ y una recta $r$; si $r$ corta a $[P,Q]$, entonces o corta a $[P,R]$ o a $[Q, R]$.

\defi{2.17 = 1.5} Una \textbf{isometría} en $\P$ es una biyección $g: \P \rightarrow \P$ que cumple que $d(g(X), g(Y)) = d(X,Y) \;\;\forall\;\; X,Y \in \P$.

\tma{2.18} Si $A,B \in \P$ y $g \in \iso(\P)$ entonces $g([A,B]) = [g(A), g(B)]$ y $g(r_{AB}) = r_{g(A)g(B)}$ 

\axioma{P5} Si $A_1, A_2 \in \P$ y $B_1, B_2 \in \P$ son dos pares de puntos que cumplen $d(A_1,A_2) = d(B_1,B_2)$ entonces existe $g \in \iso(\P)$ tal que $g(A_i) = B_i$. Se dice que esos pares de puntos son \textbf{congruentes}.

\axioma{P6} Para toda recta $r$ existe una isometría $\sigma$ llamada \textbf{reflexión} tal que  
\begin{itemizex}
	\item $\sigma(X) = X\iff X \in r$
	\item $\sigma \circ \sigma = \text{Id}$
\end{itemizex}


\defi{2.23} / \tma{2.25} / \cor{2.30} Una recta $l$ es \textbf{ortogonal} a $r$ si para todo $S \in l$ y para todo par de puntos $A, B$ que cumple que $M = \m[A,B]$, de modo que $l \cap r = M$, entonces se da que $d(A,S) = d(S,B)$. Se denota $l \perp_M r$. En estas condiciones, $l = \{X \in \P \; | \; d(S,A) = d(S,B)\}$, se denomina \textbf{mediatriz} de $[A,B]$. 

\begin{figure}[H]
	\centering
	\includegraphics[width=7cm]{figuras/2-23.png}
	\vspace{-1em}
\end{figure}

\lema{2.21} Si $\sigma_r$ entonces, para todo $X$, $\m[X, \sigma_r(X)] \in r$.

\obs{2.24} Si $l \perp r$ y $g \in \iso(\P)$ entonces $g(l) \perp g(r)$.

\importante\tma{2.26} Si $l, r \subset \P$ cortan en $M$ y $\sigma_l, \sigma_r$ son dos reflexiones de $l$ y $r$, entonces se cumple que $l \perp_M r \iff r \perp_M l \iff \sigma_r(l) = l \iff \sigma_l(r) = r$.

\importante\tma{2.27 / 2.29} Para toda recta $r$ y todo punto $S \in \P - r$, existe una recta $l$ ortogonal a $r$, que pasa por $S$. Si $r$ es una recta, y $M \in r$, entonces existe $l$ tal que $l \perp_M r$.

\axioma{P7} Para toda recta $r$ y todo punto $P$ existe sólo una recta \textbf{paralela} a $r$ que pase por $P$.

\tma{2.31 / 2.33} Si $a \perp l$ y $b \perp l$ entonces $a \parallel b$. Sean $a \parallel b$. Entonces, para todo $A \in a$, la única recta $l \perp_A a$ también es ortogonal a $b$.

\tma{2.32} Las rectas parallelas forman una relación de equivalencia.
\begin{itemizex}
	\item Reflexividad: $a\parallel a$
	\item Simetría: $a \parallel b \rightarrow b \parallel a$
	\item Transitividad  $a \parallel b $ y  $b \parallel c \rightarrow a \parallel c$
\end{itemizex}

\ej{2.6} Sean $A,B \in r$, $A \neq B$. Para todo $t$, existe un único $P_t\in r$ que cumple $d(P_t,A) = \abs{t}$ y $d(P_t, B) = \abs{t-d(A,B)}$. En definitiva, la posición de $P_t$ está sólamente determinada por las distancias $d(A, P_t)$ y $d(P_t, B)$.
	 
	 
	 
	 
	 
	 
	 \end{multicols*}\pagebreak
% multicols* obliga a terminar una columna antes de empezar la siguiente.
\DeclareMathAlphabet{\pazocal}{OMS}{zplm}{m}{n}
\let\mathcal\pazocal
\usepackage{fancyhdr}
\pagestyle{fancy}
\lhead{Alex Martínez Ascensión}
\chead{}
\rhead{\today}
\date{}
\title{\Huge\vspace{-1em}Estructuras algebráicas}
\hypersetup{
 pdfauthor={},
 pdftitle={\Huge\vspace{-1em}Estructuras algebráicas},
 pdfkeywords={},
 pdfsubject={},
 pdfcreator={Emacs 26.2 (Org mode 9.2.5)}, 
 pdflang={English}}
\begin{document}

\maketitle
\vspace{-8em}

\section*{Generalidades y teorema de Lagrange}
\label{sec:orge2ac2b7}
\subsection*{Grupos}
\label{sec:orge94b89e}
\defi{1.1} Un grupo es un conjunto no vacío \(G\) en el que se define una 
operación binaria \(G \times G \rightarrow G\;;\; (a,b) \mapsto ab\) que cumple (1) \textbf{asociatividad}
(\((ab)c = a(bc)\)), (2) \textbf{existencia de elemento neutro} \(u \in G\;;\;ua=a=au\)
y (3) \textbf{existencia de elemento inverso} \(a,x \in G \;;\; ax = u = xa\). Tanto \(u\) como \(a\) 
son únicos. Para la suma \(u = 0, a=-x\) y para el producto \(u = 1, a = x^{-1}\).

Otras propiedades inmediatas de los grupos son (1) \textbf{simplificación}:
 \(ab=ac \iff b=c\); \(ba = ca \iff b=c\); (2) \textbf{asociatividad generalizada}:
 \((a_1 \cdots a_k)(a_{k+1} \cdots a_n ) =
(a_1 \cdots a_l)(a_{l+1} \cdots a_n)\) 
, (3) \textbf{inverso de un producto}: \((a_1 \cdots a_n)^{-1}  = a_n^{-1} \cdots
a_1^{-1}\).

\defi{1.2} Un \textbf{grupo simétrico} \(S_n\) es el conjunto de biyecciones de un
conjunto \(X\) con \(n\) elementos. Se cumple que \(\text{card}(S_n) = n!\). Otros ejemplos de 
grupos son \(GL_n(\mathbb{R})\), el grupo de matrices no
singulares para la operación producto; o \(D_n\) es el conjunto de biyecciones
que conserva la distancia en un polígono de \(n\) lados. 

\defi{1.3} Un grupo es \textbf{abeliano} si \(ab = ba\;\; \forall a,b \in G\).
Todo grupo con dos elementos es abeliano, pues \(aa = aa; uu=uu; ua=a=au\);
pero para \(n \ge 3\), \(S_n\) no puede ser abeliano. \(GL_n; n \ge 2\),
ni \(D_n; n \ge 3\) son abelianos.

\importante \propo{1.4} (1) Si \(x^2 = 1 \; \forall x \in G\), entonces \(G\) es
abeliano; (2) si \((ab)^2=a^2b^2\) entonces \(G\) es abeliano.

\dem{ (1) Para cada $x,\; x \cdot x = 1 \iff x = x^{-1}$, luego si $a,b \in G$ 
entonces $a = a^{-1}; b=b^{-1}$x y si $c = ab$ entonces 
 $ab = c =c^{-1} = (ab)^{-1} = b^{-1}a^{-1} = ba$. (2) Dados $a,b \in G$,  se tiene
 que $a(ba)b = (ab)^2 = a^2b^2 = a(ab)b$ y, por simplificación, $ab = ba$.  }


\defi{1.5} Si \(G,G'\) son dos grupos con operaciones
\(G \times G \rightarrow G: (a,b) \mapsto ab \quad;\quad G' \times G' \rightarrow G': (a', b') \mapsto a'b'\)
 el \textbf{producto cartesiano} \(G'' = G \times G'\) es un grupo con operación 
\(G'' \times G'' \rightarrow G'': ((a, a'), (b, b')) = (ab, a'b')\). La asociatividad se mantiene, 
y se ve que \(1_{G''} = (1_G, 1_{G'})\).
Además, si \(G, G'\) son abelianos, \(G''\) también lo es. Se dice que \(G_1 \times \cdots \times G_r\) es el \textbf{producto directo}.

\subsection*{Subgrupos}
\label{sec:orga3e8ccf}
\defi{1.6} Un subconjunto no vacío \(H \subset G\) es un \textbf{subgrupo} de \(G\) si es un grupo con la misma operación que \(G\).
\textbf{EN ALGUNOS SITIOS \(H \subset G\) INDICO QUE \(H\) ES SUBGRUPO DE \(G\).}
Se puede ver que el elemento neutro de \(H\) es \(1_G\), y que si \(x \in H;
x^{-1} \in H\). Para que (1) \(H\) sea subgrupo de \(G\) se tiene que cumplir que
(2) si \(x,y \in H\), entonces \(x y^{-1} \in H\).

\(\{ 1_G \}\) y \(G\) son subgrupos de \(G\). El resto de subgrupos se
llaman \textbf{subgrupos propios} de \(G\). Por ejemplo, \(m \mathbb{Z} = \{ mx \;|\;
x,m \in \mathbb{Z}; 
\}\) es un subgrupo de \(\mathbb{Z}\). 

\defi{1.8.3} Se denomina a \(\langle S \rangle\) al \textbf{subgrupo generado} por \(S\). 


\(\langle S \rangle  = \left\{ s_1^{h_1} \cdots s_n^{h_n} \;|\; n \in \mathbb{N} , s_i \in S, h_i \in \mathbb{Z}
 , 1 \le i \le n
 \right\}\). Esto se puede simplificar como


\(\langle S \rangle  = \left\{ x_1 \cdots x_m \;|\; m \in \mathbb{N} , x_i \in S,  1 \le i \le m
 \right\}\). Es decir, es el conjunto de todos 
 los elementos de \(S\) combinados con operación binaria. Si \(\mathcal{F}_S\) es la
familia de los subgrupos de \(G\) que contienen a \(S\), entonces se cumple
que \(\langle S \rangle = \bigcap_{H \in \mathcal{F}_S}^{} H\).

Un caso particular es cuando \(S = \{ a \}\). En tal caso es el \textbf{subgrupo
generado por a}, \(\langle a \rangle = \{ a^k \;|\; k \in \mathbb{Z}\}\). Un subconjunto \(S \subset G\) se llama \textbf{generador de \(G\)} si \(G = \langle S \rangle\). Es cierto que
\(\langle G \rangle = G\). Si \(S\) es finito, entonces se dice que \(G\) es \textbf{finitamente generado}.

\defi{1.8.4} Si \(H\) es subgrupo de \(G\), se llama \textbf{centralizador de \(H\) en \(G\)} a \(C_G(H) = \{ x \in G \;|\; ax=xa \; \forall a \in H \}\).
El
centralizador de \(G\) en \(G\), llamado \textbf{centro de \(G\)} es el caso \(Z(G) = \{ x \in G \;|\; xa = ax \; \forall a \in G \}\). Se ve que \(C_G(H)\) es un
subgrupo de \(G\).

\defi{1.8.5}  Si \(S \subset G\) y \(a \in G\), se llama \textbf{conjugado de \(S\) por \(a\)} al conjunto 
\(S^a = \{ a^{-1}xa \;|\; x \in S \}\)

\defi{1.8.6} Si \(S \subset G\), se llama \textbf{normalizador de \(S\) en \(G\)} al conjunto 
\(N_G(S) = \{ a \in G \;|\; S^a = S \}\).
El normalizador de \(S\) es un subgrupo de \(G\) porque si \(a,b \in N_G(S)\), entonces \(S^{ab^{-1}} = (S^a)^{b^{-1}} = S^{b^{-1}} = (S^b)^{b^{-1}} =
S^{bb^{-1}} = S\)

\defi{1.8.8} Dados dos subgrupos \(K,H\) de \(G\), se define 
\(HK = \{ hk \;|\; h \in H, k \in K \}\).
Para que \(HK\) sea un subgrupo de \(G\) entonces \(HK = KH\).
Si \(H \subset K, HK = K = KH\).

\subsection*{Orden de un grupo}
\label{sec:org84e3a9c}
\defi{1.9} El \textbf{orden} de un subgrupo finito \(H \subset G\) es el número de elementos que tiene. Se denota por \(o(H)\). Un elemento \(a \in G\) es \textbf{de torsión} si \(\langle a \rangle\) es finito. En tal caso el orden
es \(o(a)\).

\importante\propo{1.10} Sea \(G\) un grupo y \(a \in G\) de torsión. Entonces se cumple
que 
\begin{itemize}
\item Existe \(k \ge 1\) tal que \(a^k=1\)
\item \(o(a)\) es el menor número tal que \(a^n=1\)
\item Si \(n=o(a), \langle a \rangle = \{ 1, a, \cdots, a^{n-1} \}\)
\item \(a^k=1\) sii \(k\) es múltiplo de \(n\)
\item \(o(a^{-1}) = o(a)\)
\item Si \(x = a^k\) y \(n=o(a)\), entonces \(o(x) = \frac{n}{mcd(n,k)}\)
\item Si \(b \in G\) es de torsion y \(ab=ba\) entonces \(o(ab)\) es divisor de
\(mcm(o(a),o(b))\). Si \(o(a), o(b)\) son primos entre si, \(o(ab) =
  o(a)o(b)\)
\item \(o(ab) = o(ba)\)
\end{itemize}

\subsection*{Índice de un subgrupo}
\label{sec:orgd335343}
\defi{1.2}/\obs{1.12.6/7} Sea \(G\) un grupo y \(H \subset G\).  Sean \(R^H, R_H\) las relaciones de equivalencia en \(G\):
\[ xR_H y \iff xy^{-1} \in H \]
\[ xR^H y \iff x^{-1}y \in H \]
Además, se definen \(Hx = \{ hx \;|\; h \in H \}; xH = \{ xh \;|\; h \in H \}\).
Se cumple que si \(x,y \in G\), y \(yR_H x\) entonces \(yx^{-1} = h \in H\) y,
por tanto, \(y = hx \in Hx\).

Además, las aplicaciones \(H \rightarrow Hx: h \mapsto hx\) y su equivalente en \(xH\) son
biyectivas. Es importante que pese a existir una biyeccíón entre
 \(Hx\) y \(xH\), \textbf{no siempre \(Hx\) y \(xH\) son iguales}.

\propo{1.12.3} La aplicación entre conjuntos cocientes
 \(G/R_H \rightarrow G/R^H: Hx \rightarrow x^{-1}H\) es biyectiva.

\defi{1.12.4} \(H \subset G\) es un subgrupo de \textbf{índice infinito} si \(G/R_H\) es
un conjunto infinito. Por otra parte, el índice de \(H\) en \(G\), \([G:H]\) es
 el número de elementos de \(G/R_H\).

\importante \propo{1.12.8 (T de Lagrange)} Sea \(H \subset G\) un subgrupo. Se cumple que 
si \(G\) es finito, entonces \(o(H)\) es
finito, \(H\) tiene índice finito en \(G\) y \(o(G) = o(H) \cdot [G:H]\).

\cor{1.12.9} Si \(H,K\) son subgrupos finitos de \(G\), \(o(H) = m\), y 
mcd(\(m, n) = 1\) entonces \(H \cap K = \{ 1_G \}\).

\propo{1.12.10 (F de transitividad del índice)} Sean \(H,K\) 
subgrupos de \(G\). Si \(H\) es subgrupo de \(K\), y los indices entre subgrupos, y con G,
son finitos, entonces se cumple \([G:K] = [G:H][H:K]\)

\importante\propo{1.12.11} Sean \(H,K\) subgrupos de \(G\), finito. Entonces
\[ card(HK) = \frac{o(H)o(K)}{o(H \cap K)} \]


\importante  \defi{1.15} / \obs{1.15.4} Un grupo \(G\) se llama \textbf{cíclico} si existe
 \(a \in G\) tal que \(G
= \langle a \rangle\). Si \(o(a) = p\), primo, el grupo es cíclico.

\importante\propo{1.16 / 1.17} Sea \(G\) cíclico y \(n = o(G)\), para cada divisor \(m\) de
\(n\)
existe un único subgrupo de \(G\) de orden \(m\), y ese subgrupo es cíclico.
Además, todo subgrupo de un grupo cíclico [finito o no] es cíclico.

\defi{1.18} Sea \(G\) finitamente generado. Un sistema generador \(S\) se
llama \textbf{minimal} si cualquier subconjunto de \(G\) con menos elemenos que \(S\)
no es generador de \(G\).

\propo{1.19} Sea \(G\) finito de orden \(n\) y \(S = \{ x_1, \cdots, x_p \}\)
un
sistema generador minimal de \(G\). Entonces \(2^p \le n\).

\dem{ Llamamos \( S_i = \{ x_1, \cdots, x_i \}, \; 1 \le i \le p \); y \( H_i = \langle S_i \rangle \). 
Evidentemente, \( H_i \subset H_{i+1}
\). Por ele teorema de Lagrange y la fórmula de la transitividad del índice,
\[ [G:H_1] = [H_P:H_1] = [H_P:H_{P-1}][H_{P-1}:H_{P-2}]\cdots[H_2:H_1] \]
Además, 
\[ [H_{i+1}:H_i] = \frac{o(H_{i+1})}{o(H_i)} > 1 \iff
[H_{i+1}:H_i] \ge 2 \]
pues los índices son enteros.
Por tanto, \( [G:H_1] \ge 2^{p-1} \), y como \( o(H_1) \ge 2 \), entonces,
\( o(G) = o(H_1)[G:H_1] \ge 2^p \)
}

\pagebreak

\defi{Grupo diédrico \(D_n\)} Definimos el grupo diédrico al grupo con las
operaciones \(f, g\) tales que \(o(g) = 2\) y \(o(f) = n\). Entonces \(D_n = \{ 1, f, f^2, \cdots, f^{n-1}, g, gf, \cdots, gf^{n-1}  \}\). Este grupo
posee unas propiedades demostrables:
\begin{itemize}
\item \(f^kgf^k = g\)
\item \(gf^kg = f^{n-k}\)
\end{itemize}

\defi{Grupo cuaternión \( Q \)} El grupo cuaternión tiene los elementos \(Q =
\{ 1, -1, i, j, k, -i, -j, -k \}\). Las relaciones que definen el grupo son \(i^2 = j^2 = k^2 = ijk = -1\).


\pagebreak


\section*{Subgrupos normales, homomorfismos, teorema de estructura de grupos abelianos finitos}
\label{sec:org8dcecf7}
\importante  \propo{2.1} Sea \(G\) un grupo y \(H\) un subgrupo. \(H\) es un subgrupo
 \textbf{normal} (\textbf{LO DEFINO AQUÍ COMO} \(\subset_N\)) si se cumplen las condiciones equivalentes:
\begin{enumerate}
\item Para todo \(a  \in G\), \(aH = Ha\)
\item Para todo \(a \in G\), \(H = H^a\)
\item Para todo \(a,b \in G\), \(ab \in H \iff ba \in H\), luego si \(H\) es
abeliano, es normal.
\end{enumerate}

\dem{ 1 \(\implies\) 2. Si \(y \in H^a\) entonces \( aya ^{-1} = h \in H  \). Como
\( ay = ha \in Ha = aH \) existe  \( h' \in H \) tal que \( ay = ah' \). 
Así, \( y \in H^a = h' \in H \iff H^a \subset H \). Si aplicamos lo mismo con \( xa ^{-1}  \)
 tenemos \( H^{a ^{-1}} \subset H
\) y, con ello \( H \subset H^a
\), luego \( H = H^a \).
2 \(\implies\) 3. Como \( ab \in H \), \( ba = a^{-1}aba \in H^a \), y como \( ba \in H \), \( H = H^a \).
3 \(\implies\) 1. Sea \( x \in Ha \), luego \( \exists h \in H, x = ha \), y \( xa ^{-1} = h \in H \). 
Por hipótesis \( h' = a ^{-1}x \in H \) y \( x = ah' \in aH \), luego \( Ha \subset aH
\). Si empezamos con \( x \in aH \) obtenemos que \( aH \subset Ha
\), luego \( aH = Ha \). }


\obs{2.2.1} Si \(H\) es normal, entonces \(R^H = R_H\), y \(G/R_H\) se
escribe \(G/H\).

\importante\obs{2.2.4/2.2.5} Si \(H\) es un subgrupo de \(G\), y \([G:H] = 2\), \(H\) es subgrupo normal de \(G\). Asimismo, los subgrupos \(\{ 1_G \}, G\) son 
normales.

\importante\defi{2.2.14} Un grupo \(G\) es \textbf{simple} si los únicos subgrupos son  \(\{ 1_G
\}, G\). Si \(o(G)\) es primo \(p\), por el teorema de Lagrange, los únicos
subgrupos son \(\{ 1_G \}, G\), luego \(G\) es simple.

\propo{2.2.8} Todo subgrupo \(H \subset Z(G) = \{ a \in G \;|\; ag = ga\;
\forall g \in
G\}\) es subgrupo normal de \(G\).

\propo{2.2.10} Sea \(H\) subgrupo de \(G\).
\begin{enumerate}
\item \(H\) es subgrupo de \(N_G(H) = \{ a \in G \;|\; H = H^a \}\).
\item \(H \subset_N
   N_G(H)\).
\item Si \(H \subset K \subset G\) y \(H \subset_N
   K\),
entonces \(K \subset N_G(H)\).
\end{enumerate}

\defi{2.2.11} Si \(H,K\) son subgrupos de \(G\), \(K\) es un \textbf{subgrupo
conjugado} de \(H\) si existe \(a \in G\) tal que \(K = H^a\). Como la
relación es recíproca, se dice que \(K\) y \(H\) son conjugados.

\propo{2.2.11} 
\begin{itemize}
\item Si \(\Sigma\) es la familia de conjugados de \(H\) y \(N = N_G(H)\), la aplicación \(\phi: G/R_N \rightarrow \Sigma: Na \rightarrow H^a\) es biyectiva.
\item Si \(N\) tiene índice finito en \(G\), el número de conjugados con \(H\) es \([G:N]\).
\end{itemize}

\propo{2.2.13} Si \(A \subset_N G\), \(H \subset K \subset G\), y \(H \subset_N K\), entonces \(AH \subset_N AK\).

\importante La normalidad no es transitiva, es decir, si \(H \subset_N
K \subset_N G\), no siempre es cierto que \(H \subset_N G\).

\defi{2.2.16}/\obs{2.2.16.1} Si \(H \subset G\), se llama corazón de \(H\) a \[
\heartsuit(H) =  K(H) = \bigcap_{a \in G}^{} H^a \]

Si \(N \subset_N H\) entonces \(N \subset K(H)\), pues para cada \(a \in G\): \(N = N^a
\subset H^a\), luego \(N = N^a \subset \cap_{a \in G} H^a = K(H)\)

\propo{2.2.17 (T de Poincaré)} Si \(G\) posee un subgrupo de índice
finito, también posee un subgrupo normal de índice finito.

\subsection*{Grupos cocientes}
\label{sec:org36de290}
\propo{2.3} El grupo cociente \(G/H\) de \(H \subset_N G\) tiene estructura de grupo con la operación:
\begin{align*}
 &\; G/H \times G/H  
 \longrightarrow G/H 
 \\
$       &\; (aH, bH) \longmapsto     abH 
\end{align*}

El elemento neutro del grupo cociente es \(H\), y el inverso de \(aH\) es \((aH)^{-1} = a^{-1}H\).

\obs{2.3.1} Si \(H \subset_N K \subset G\) (entonces \(H \subset_N G\)), el
grupo cociente \(K/H \subset G/H\),  ya que si \(aH,bH \in K/H, a,b \in K\), entonces \((aH)(bH)^{-1} =
(aH)(b^{-1}H) = ab^{-1}H \in K/H\), ya que como \(K \subset G
, ab^{-1} \in K\)

\obs{2.3.1.1} \(K \subset_N G \iff K/H \subset_N G/H\)

\ejem{2.3.3 (Función \( \phi \) de Euler)} Si denotamos \(\mathbb{Z}^*_m = \{
a+m \mathbb{Z} \in \mathbb{Z} / m \mathbb{Z}
\;|\; mcd(a,m) = 1 \}\) y consideramos la operación binaria
\[ \mathbb{Z}^*_m \times \mathbb{Z}^*_m \rightarrow \mathbb{Z}^*_m: (a+m \mathbb{Z},
b+ m \mathbb{Z}) \mapsto ab + m \mathbb{Z}
\]
vemos que \(\mathbb{Z}^*_m\) forma un grupo abeliano con elemento neutro \(1 +
m \mathbb{Z}\) y elemento inverso \(u + m \mathbb{Z}\), con \(au = 1\).

Entonces, la función \(\phi: \mathbb{N}\{ 0 \} \rightarrow  \mathbb{N}\{ 0 \}\) que a cada \(m\) positivo le corresponde el orden de \(\mathbb{Z}^*_m\) es
la función de Euler. Para \(p\) primo, \(\phi(p) = p-1\), y \(\phi(p^k) = p^{k-1}(p-1)\). Si
tenemos \(m,n\) tal que \(mcd(m,n)=1\), entonces \(\phi(mn) =
\phi(m)\phi(n)\). Con todo esto, si tenemos un numero \(a = p_1^{k_1} \cdots
p_i^{k_i}\), entonces \(\phi(a) = p_1^{k_1-1} \cdots p_i^{k_i-1} (p_1-1) \cdots(p_i-1)\)

\subsection*{Homomorfismos}
\label{sec:org82943c2}
 \importante  \defi{2.4 / 2.6} Una aplicación \(f:G \rightarrow G'\) es un \textbf{homomorfismo de grupos} si \(f(ab) = f(a)f(b) \forall a,b \in G\).
Para todo homomorfismo se tiene que \(f(1_G) = 1_{G'}\) y \(f(a^{-1}) =
(f(a))^{-1}\). Si un homomorfismo es biyectivo se llama \textbf{isomorfismo}. 
Se denota por \(G \simeq G'\) cuando dos grupos son isomorfos.

\defi{2.4.3}/\propo{2.4.4} El \textbf{núcleo} de un homomorfismo es \(ker(f) = \{ a \in G \;|\; f(a) =
1_{G'} \}\). \(f\) es inyectiva sii \(ker(f) = \{ 1_G \}\).

\defi{2.4.5} Se llama \textbf{imagen} de \(f\) a \(im(f) = \{ f(x) \;|\; x \in G \}\).

\propo{2.4.6} Si \(f: G \rightarrow G'\) es homomorfismo y \(H' \subset G'\), entonces \(f^{-1}(H') = \{ x \in G \;|\; f(x) \in H' \}\) es un subgrupo
de \(G\). Además, si \(H' \subset_N G'\) entonces \(f^{-1}(H') \subset_N G\).

\obs{2.4.7/2.4.8} Si \(H \subset G\), la inclusión \(j: H \rightarrow  G: x \mapsto x\) es un homomorfismo inyectivo; y si \(H \subset_N G\), la proyección \(\pi: G \rightarrow  G/H : x \mapsto xH\) es un homomorfismo sobreyectivo.

\propo{2.4.9} Si \(f: G \rightarrow G'\) y \(g: G' \rightarrow G''\) son homomorfismos, \(g \circ f: G \rightarrow G''\) también lo es, pues \((g \circ f)(xy) = g(f(xy)) = g(f(x)f(y)) =
g(f(x))g(f(y)) = (g \circ f)(x) (g \circ f)(y)\)

\propo{2.4.10} Si \(f\) es un homomorfismo y \(x \in G\) tiene orden \(m\), se cumple que (i) \(o(f(x))\) divide a \(m\)  (ii) Si \(f\) es
inyectiva, \(o(f(x)) = m\).

\importante \propo{2.5 (Factorización canónica de un homomorfismo)}

Sea \(f: G \rightarrow G\) un homomorfismo. Entonces existe un homomorfismo biyectivo \(b: G/ker(f) \rightarrow im(f)\) que hace conmutativo el diagrama

\vspace{-2em}
\begin{center}
\begin{tikzpicture}[]
\node (a) at (0,2*0.8) {$G$};
\node (b) at (0,0) {$G/ker(f)$};
\node (c) at (4*0.8,0) {$im(f)$};
\node (d) at (4*0.8,2*0.8) {$G'$};

\draw[->,thick] (a) -- node[above] {$f$} (d);
\draw[->,thick] (a) -- node[left] {$\pi$} (b);
\draw[->,thick] (b) -- node[below] {$b$} (c);
\draw[->,thick] (c) -- node[right] {$j$} (d);
\end{tikzpicture}
\end{center}
\vspace{-2em}

\dem{ \(f(x) = j((b \circ pi) (x)) = b(\pi(x)) = b(x\; ker(f))\). Comprobamos que se cumple el enunciado. 
(1) \( b \) está bien definida porque si \( x\; ker(f) = y\; ker(f) \) entonces 
\( x ^{-1}y \in ker(f)\;,\; f(x) ^{-1}f(y) = 1_{G'} \iff f(y) = f(x)
\), luego \( b(x \; ker(f)) = b(y \; ker(f)) \). (2) \( b \) es homomorfismo, ya que 
\( b((x \, ker(f))(y \, ker(f)) = b(xy \, ker(f) = f(xy) = f(x)f(y) =
b(x \, ker(f))b(y \, ker(f)) \). (3) \( b \) es inyectivo, ya que si \( x \, ker(f) \in ker(b) \)
entonces \( f(x) = b(x \, ker(f)) = 1_{im(f)} = 1_{G'} \), y \( x in ker(f) \).
(4) \( b \) es sobreyectiva ya que para cada elemento de \( im(f) \) existe un 
elemento de la forma \( b(x \, ker(f) \).  }


\propo{2.6.X (Propiedades de isomorfismos)} Si \(G \simeq G'\), y \(G\) es
abeliano o cíclico, entonces \(G'\) también lo es. Si \(X,Y\) son conjuntos
con el mismo número de elementos, entonces \(Biy(X) \simeq Biy(Y)\).

\importante \cor{2.7 (Primer teorema de la isomorfía)} Si \(f: G \rightarrow  G'\) es un homomorfismo, \(G/ker(f) \simeq im(f)\).

\cor{2.8} Todo grupo cíclico es isomorfo a \(\mathbb{Z}\) o a \(\mathbb{Z}/m \mathbb{Z}.\)
\dem{ Sea \( G = \langle a \rangle  \) cíclico. Consideramos \( f: \mathbb{Z} \rightarrow G: k \rightarrow f(k) = a^k
\). Como \( f(x+y) = a^{x+y} = a^xa^y = f(x)f(y) \), \( f \) es homomorfismo. 
Cada elemento \( b \in G \) es de la forma \( a^k \), luego \( f \) es sobreyectiva, 
es decir, \( im(f) = G \). Por el primer tma de isomorfía tenemos que \( \mathbb{Z}/ker(f) \simeq im(f) \),
luego \( \mathbb{Z}/ker(f) \simeq G \), y como \( ker(f) \) es subgrupo de \( \mathbb{Z}\), existe \( m \) 
tal que \( ker(f) = m \mathbb{Z} \). Si \( m = 0 \), \( ker(f) = 0 \mathbb{Z} = \{ 0 \}
\), y \( \mathbb{Z} \simeq G
\). Si \( m > 0 \), \( \mathbb{Z}/m \mathbb{Z} \simeq G \).}

\ejem{2.9.4.4} Para \(f: \mathbb{Z}/m \mathbb{Z} \rightarrow \mathbb{Z}/n \mathbb{Z}\), si \(n\) es múltiplo de \(m\), existen homomorfismos
inyectivos y el número de homomorfismos es \(\phi(m)\). Si \(mcd(m,n) = n\),
existen isomorfismos sobreyectivos y el número es \(\phi(n)\). El número de homomorfismos en general es 
\(\text{mcd}(n,m)\).

\ejem{2.9.6} Sea \(n \ge 2\) y \(X = \{ 1, 2, \cdots, n \}\) y \(f_n \in
S_n = Biy(X)\). Se llama \textbf{signatura de f}, \(s(f)\) al número de pares \((i,j)
\in X \times X\) tales que \(i < j\;\;\; f(i) > (j)\).
La aplicación \(\epsilon: S_n \rightarrow U_2 = \{ -1,1 \}\;\; f \mapsto
\epsilon (f) = (-1)^{s(f)}\)
es homomorfismo. Esta fórmula puede ser calculada también así:
\(\epsilon (f) = \prod_{i < j}^{} \frac{f(i) - f(j)}{i-j}\)
Se denomina \textbf{grupo alternado}, \(A_n\) al núcleo de \(\epsilon\): \(A_n = \{
f \in S_n \;|\; \epsilon (f) = 1\}\)

\propo{2.10} Si \(G\) es un grupo con \(o(G) < 12\), para cada dividor \(d\) de \(n\) existe un subgrupo \(G, o(G) = d\). Sin embargo, si \(o(G) \ge
12\), no siempre se cumple esto (\(A_4\) tiene \(o(A_4) = 12\), pero no
tiene subgrupos de orden \(6\). Esto verifica que el \textbf{recíproco del teorema de
Lagrange no es cierto siempre}. 

\subsection*{Teoremas de isomorfía}
\label{sec:org7200f6f}
\propo{2.15 (Segundo teorema de isomorfía)} Sean \(N,H \subset_N G\), y \(N \subset H\). Entonces \(H/N \subset_N G/N\) y \((G/N)/(H/N) \simeq G/H\)

\propo{2.16 (Tercer teorema de isomorfía} Si \(H,N \subset G\), y \(N \subset_N G\), 
\begin{enumerate}
\item \(H \cap N \subset_N H\)
\item \(HN \subset G\)
\item \(N \subset_N HN\)
\item \(HN/N \simeq H/(H \cap N)\)
\end{enumerate}

\lema{2.17} Sean \(A,B,C \subset G\), y \(B \subset A\). Entonces \(A \cap BC = B(A \cap C)\)

\propo{2.18 (Cuarto teorema de isomorfía)}. Sea \(H_1, H_2 \subset G\), \(N_i \subset_N H_i\). Entonces
\begin{itemize}
\item \(N_1(H_1 \cap H_2) \subset H_1\) y \(N_2(H_1 \cap H_2) \subset H_2\)
\item \(N_1(H_1 \cap N_2) \subset_N   N_1(H_1 \cap H_2)\) y 
\(N_2(N_1 \cap H_2) \subset_N   N_2(H_1 \cap H_2)\)
\item \((H_1 \cap N_2)(N_1 \cap H_2) \subset_N H_1 \cap H_2\)
\item \((N_1(H_1 \cap H_2))/(N_1(H_1 \cap N_2)) \simeq (H_1 \cap H_2)/(H_1 \cap
  N_2)\;(N_1 \cap H_2) \simeq (N_2(H_1 \cap H_2))/(N_2(N_1 \cap H_2)\)
\end{itemize}

\subsection*{Estructura de grupos abelianos finitos}
\label{sec:org01e4e7e}
\lema{2.20} Sea \(G\) grupo abeliano finito y \(x \in G\) un elemento de
orden máximo. Entonces, para cada \(y \in G\), el orden de \(y\) divide al
de \(x\).

\lema{2.20.2} Sea \(G\) abeliano finito y \(x \in G\) de orden máximo. Sean
\(H = \langle x \rangle\) e \(y \in G\), entonces existe \(z \in Hy\) tal
que \(o(z) = o(Hy)\)

\importante\lema{2.20.3} Sean \(H,K \subset_N G\) tales que \(H \cap K = \{ 1 \}\). Entonces \(HK \simeq H \times K\).

\importante \propo{2.2.1 (Teorema de estructura de grupos abelianos finitos)} Si \(G\) es
abeliano finito, existen \(m_1, \cdots, m_r\), denominados \textbf{coeficientes de
torsión de G}, tales que \[ G \simeq \mathbb{Z}/m_1 \mathbb{Z} \times \cdots
\times \mathbb{Z}/m_r \mathbb{Z}\] y cada \(m_i\) divide a \(m_{i-1}\).
Además, los coeficientes son únicos.

\importante\propo{2.22} Sean \(G_1, G_2\) grupos cíclicos de órdenes \(m\) y \(n\), y 
\(G = G_1 \times G_2\), entonces las afirmaciones (1) \(G\) es cíclico, (2) \(mcd(m,n) = 1\) son
equivalentes.

\dem{ (1) $\rightarrow$ (2). Sea \( M \) el \( mcm(m,n) \). Escribimos \( M=ma, M=nb \).
 Cada \( u=(x,y) \in G \) cumple \( u^M = (x^M, y^M) = ((x^m)^a, (y^n)^b) = (1_{G_1}^a, 1_{G_2}^b) = 1_G \)
 Como \( G \) es cíclico y \( o(G) = m \), alguno de sus elementos tiene orden \( mn \). Luego \( mn=N \) y \( mcd(m,n) = 1\)   }


\obs{2.22.3} Si \(p_1 < \cdots < p_s\) son primos, todo grupo abeliano de
orden \(n = p_1 \cdots p_s\) es cíclico.

\propo{2.23} Si \(p\) es primo, \(\mathbb{Z}_p^*\) es cíclico.

\section*{Grupos de automorfismos. Acción de un grupo sobre un conjunto}
\label{sec:org214f481}
\cor{3.4.3.1} Sea \(G\) un grupo y \(H \subset G\), \(H \subset Z(G)\).
Entonces (1) \(H \subset_N G\) y (2) Si \(G/H\) es cíclico, \(G\) es
abeliano.  
\subsection*{Acciones de grupos sobre conjuntos}
\label{sec:orgeae3d97}
\defi{3.8} Una \textbf{acción sobre un conjunto} \(X\) es la aplicación \(G \times
   X \rightarrow X: (g,x) \mapsto g(x) \forall g \in G, \forall x \in X\). Se
cumple (1) \((gh)(x) = g(h(x))\), (2) \(1_G(x) = x\).

\obs{3.8.3} El homomorfismo \(\theta: \; G \longrightarrow Biy(Y) \;;\; g
   \mapsto \theta(g): X \rightarrow X: x \mapsto g(x)\) es la que define \(g\) como acción sobre \(X\). Definimos \(ker(\theta) = \{ g \in G | g(x) = x \forall x \in X \}\). Se dice que una
acción es \textbf{fiel} cuando \(ker (\theta) = \{ 1_G \}\).

\importante\propo{3.9 (Teorema de Cayley)} Todo grupo \(G\) es isomorfo a un subgrupo \(Biy(G)\).

\propo{3.10} Sea \(G \times X \rightarrow X\) una acción. Se llaman
\begin{itemize}
\item \textbf{Estabilizador} de \(x\): \(G_x = \{ g \in G| g(x) = x \}\)
\item \textbf{Órbita} de \(x\): \(O_x = \{ g(x) | g \in G \} \subset X\). \(\cup O_x = X\) y \(\cap
  O_x = \emptyset\).
\item \(x\) es un \textbf{punto fijo} de \(g\) si \(G_x = G\).
\end{itemize}

\defi{} Una acción es \textbf{libre} si \(G_x = \{ 1_G \}\). Una acción es \textbf{efectiva} si 
\(\cap_{x \in X} G_x = \{ 1_G \}\).

\obs{3.10.3} La aplicación \(G/R^{G_x} \rightarrow O_x: gG_x \mapsto g(x)\) es biyectiva, y \(card(O_x) = [G:G_x]\). 

\obs{3.10.4} \(ker(\theta) = \bigcap_{x \in X} {G_x}\)

\defi{3.11.2.1} Se llama \textbf{clase de conjugación del elemento \(x\)} a \(Cl(x) =
\{ axa^{-1} | a \in G \}\), y el conjunto de elementos de órbitas (un elemento
por órbita), \(Y\) se llama \textbf{sistema de representantes por conjugación de \(G\)}.

\defi{} \(Z(G) = \{ a \in G | ax = xa \; \forall x \in G \}\), \(C_G(x) = \{ a \in
G | ax = xa \}\).

\importante\propo{3.11.3} Sea \(G\) finito e \(Y\) un sistema de representantes por
conjugación. Entonces \(o(G) = o(Z(G)) + \sum _{x \in Y\;;\;x \not\in Z(G)}^{} [G:C_G(x)]\)

\cor{3.11.4} Sea \(p\) primo, \(m \ge 1 \in \mathbb{N}\) y \(G\) de orden \(p^m\).
Entonces \(Z(G) \neq \{1_G\}\) y \(o(Z(G)) \neq p^{m-1}\).

\cor{3.11.5} Sea \(p\) primo y \(G\) de orden \(p^2\). Entonces \(G\) es 
abeliano e isomorfo a \(\mathbb{Z}/p^2 \mathbb{Z}\) o a \(\mathbb{Z}/p \mathbb{Z} \times  \mathbb{Z}/p \mathbb{Z}\).

\cor{3.11.6} Sea \(G\) de orden \(p^m, m > 1\). Entonces \(G\) no es
simple.
\dem{ Si \( G \) es abeliano, \( Z(G) \subset_N G \) ya que \( \{ 1 \} \neq_{3.11.5} Z(G) \) y \( Z(G) \neq G \) 
porque si no es abeliano.  
Si \( G \) es abeliano tomamos \( a \in G/\{1\} \); y si \( o(a) \neq p^m \), \( H = \langle a \rangle \)
es grupo normal y \( \{1\} \neq H \neq G \).
Si \( o(a) = p^m \), tomamos \( b = a^p \in G \). Entonces \( o(b) = \frac{p^m}{mcd(p^m,p)} = p^{m-1} \) y 
\( H = \langle b \rangle \) es subgrupo normal}

\lema{3.11.8} Sea \(G\) finito, y \(a,b \in G\) elementos de orden 2, tal que
\(a \not\in Cl(b)\); entonces existe \(c\) de orden 2 de \(G\) tal que \(a,b \in
G\).

\propo{3.11.9 (Teorema de Brauer)} Sea \(G\) finito de orden par con \(a,b \in
G\) y \(a \not\in Cl(b)\). Sea \(m\) el máximo de los órdenes de \(C_G(x)\);
entonces \(o(G) < m^3\). 

\ejem{3.12} Sean \(H \subset G\) y \(X = G/R^H\). Definimos la acción \(G
\times G/R^H \rightarrow G/R^H \;;\; g(xH) = gxH\). Entonces \(ker(\theta) = K(H)\).

\propo{3.12.1} (generaliza 2.2.4). Sea \(G\) finito y \(p \neq o(G)\) el
menor divisor de \(G\), y \(H \subset G, o(H) = p\). Entonces \(H \subset_N
G\) y \(G\) no es simple. 

\propo{3.12.2} Sea \(G\) finito de orden \(n\) y \(H \subset G\), \([G:H]
= m \neq 1\). Si \(n  \nmid m!\)  entonces \(K(H) \neq \{ 1 \}\) y \(G\) no es
simple. 

\importante\propo{3.12.3} Sea \(G\) finito con \(o(G) > 2\). Si \(G\) posee un
subgrupo \(H\), \([G:H] = n \neq 1\), y \(G \not \simeq A_n\), entonces \(G\) no es simple.

\ejem{3.12.3.1} Si \(o(G) \ge 5\), y \(H \subset G\) tal que \(2 \le [G:H]
\le 4\), entonces \(G\) no es simple.

\propo{3.12.4} Sea \(G\) finítamente generado. Para cada \(n\), \(G\)
tiene una cantidad finita (o nula) de subgrupos de índice \(n\).

\importante \cor{3.12.5 (Generalización Teorema de Poincaré)}
Si \(G\) es finitamente generado y \(H \subset G\) tiene índice finito,
existe un subgrupo \(K \subset H\) tal que \([G:K]\) es finito.

\importante \propo{3.13 (Teorema de Cauchy-Fröbenius)} Sea \(p\) primo y \(G\) finito
con orden múltiplo de \(p\). Entonces el número de elementos \(y \in G\)
tales que \(y^p=1\) es múltiplo de \(p\). Existe un elemento de orden \(p\) en \(G\).

\cor{3.14} Sean \(m,p\), \(p\) primo tales que \(p >m>1\). Los grupos de
orden \(mp\) no son simples.

\cor{3.15} Sean \(p,q\) primos. Entonces, un grupo de orden \(pq\) no es
simple. 

\cor{3.16} Sea \(G\) con \(o(G) > 1\). Entonces \(G\) es simple sii \(G\) es finito y \(o(G)\) es primo.

\propo{3.17} Sea \(G\) un grupo de orden \(p^m\), \(p\) primo y \(m \ge 1\). Si \(H \neq \{ 1 \} \subset_N G\), entonces \(H \cap Z(G) \neq \{ 1 \}\).

\cor{3.18}  Sea \(p\) primo, \(m \ge 1\) uy \(o(G) = p^m\), y \(H\)
subgrupo normal de orden \(p\), entonces \(H \subset Z(G)\).

\ejem{3.19} El grupo cuaternión \(Q\) y el diedral \(D_4\) son, salvo
isomorfismo, los únicos grupos no abelianos de orden 8.





\pagebreak

\section*{Grupos abelianos finitamente generados. Generadores relacionales}
\label{sec:org2b4280e}
\subsection*{Teorema de la Estructura}
\label{sec:orgff4f93a}
\importante \defi{7.1} Si \(G\) es un grupo, denotamos \(T(G) = \{ x \in G \;|\; o(x)
   \text{ es finito} \}\). \(G\) tiene \textbf{torsión} si \(T(G) \neq \{ 1 \}\).
\begin{enumerate}
\item Si \(G\) es abeliano, \(T(G) \subset G\) ya que \(o(1) = 1\), y si \(x,y \in T(G)\), \(o(xy ^{-1}) |
   mcm(o(x),o(y))\), luego \(o(xy) = o(xy ^{-1})\) es finito y \(xy ^{-1} \in
   T(G)\).
\item El cociente \(G/T(G)\) no tiene torsión.
\item Si \(G\) no es abeliano, \(T(G)\) no tiene por qué ser subgrupo de \(G\).
\item El grupo \(G = \mathbb{Z} \times \cdots \times \mathbb{Z}\) no tiene torsión
\item Si \(G\) es finito, \(G = T(G)\) ya que \(\forall x \in G, x^{O(G)} = 1\).
\item Si \(G\) es infinito, puede ser que \(G = T(G)\).
\item Si \(G\) es abeliano y \(K \subset G\), \(T(K) = K \cap T(G)\). Si \(K\) no tiene torsión, \(K \cap T(G) = \{
   1 \}\), y si \(G\) no tiene torsión, \(K\) tampoco la tiene.
\item Si \(K = T(G)\), \(T(T(G)) = T(K) =  K \cap T(G) = T(G)\)
\end{enumerate}

\importante\ejem{7.2 / 7.3} Si \(G = \mathbb{Z}/m_1 \mathbb{Z} \times \cdots \times
\mathbb{Z}/m_r \mathbb{Z} \times \mathbb{Z} \times \cdots \times \mathbb{Z}\), entonces \(T(G) = \mathbb{Z}/m_1 \mathbb{Z} \times \cdots \times
\mathbb{Z}/m_r \mathbb{Z} \times \{0\} \times \cdots \times \{0\}\) y \(G/T(G) =\mathbb{Z} \times \cdots \times \mathbb{Z}\)

\lema{7.5/7.6} Sea \(\{ x_1, \cdots, x_k \}\) un sistema generador minimal de \(G\). Si existen enteros no nulos \(m_1, \cdots, m_k\), tales que \(\prod
x_i^{m_i} = 1\), entonces \(G\) tiene torsión. Sea \(G\) abeliano, sin torsión y finitamente generado. Si \(n\)
es el mínimo número de generadores de \(G\), existen subgrupos de \(G\) cíclicos \(H_1, \cdots, H_n \simeq \mathbb{Z}\) tales que \(G = H_1 \cdots H_n \simeq H_1 \times \cdots \times H_n\) 

\importante\propo{7.7 (Teorema de estructura de grupos abelianos finitamente generados)}
Sea \(G\) abeliano finitamente generado. Existen enteros no negativos \(n,r\), y si \(n \neq 0\), enteros positivos \(m_1, \cdots, m_n\), todos únicos,
 tales que 
\begin{enumerate}
\item \(G \simeq  \mathbb{Z}/m_1 \mathbb{Z} \times \cdots \times
    \mathbb{Z}/m_n \mathbb{Z} \times \mathbb{Z} \times \cdots_r \times \mathbb{Z}\)
\item \(m_i\) divide a \(m_{i-1}\) para cada \(2 \le i \le n\).
\item \(T(G) \simeq  \mathbb{Z}/m_1 \mathbb{Z} \times \cdots \times
   \mathbb{Z}/m_n\) y \(r = \beta(G) =\) número de Betti de \(G\).
\item \(G \simeq T(G) \times G/T(G)\).
\end{enumerate}

\importante\ejem{7.8} Si \(G\) no es finítamente generado, puede ser que \(G \not\simeq
T(G) \times G/T(G)\). P. ej. en el grupo \(G = \prod_{n \in \mathbb{N}}^{}
\mathbb{Z}/p_n \mathbb{Z}\), si \(T(G) = \{ a \in G \;|\; \text{sop}(a) \text{ es finito} \}\), \(\text{sop}(a) = \{ n \in \mathbb{N} \;|\; a_n + p_n \mathbb{Z} \neq 0 + p_n \mathbb{Z}
\}\) falla en cumplir la isomorfía.

\ejer{123} Un grupo abeliano \(G\) es \textbf{libre} si existe algún conjunto \(I\) no vacío 
tal que \(G \simeq \mathbb{Z}^{(I)}\). [Wikipedia: si todo elemento de \(G\) puede
escribirse de forma única como producto de finitos elementos de \(I\) y sus inversos.


\subsection*{Generadores y relaciones}
\label{sec:org4ccd118}
   \propo{7.10} Sea \(G\) abeliano y finitamente generado, con \(S = \{ x_1,
   \cdots, x_n \}\) un sistema de generadores. Sea \(f_S\) el homomorfismo 
\(f_S: \mathbb{Z} \times  \cdots_n \times  \mathbb{Z} = \mathbb{Z}^n \rightarrow G: (m_1, \cdots, m_n) \mapsto m_1 x_1+ \cdots+
   m_nx_n\) y sea \(R(S)\) = \(ker(f_s)\).
Vemos que \(ker(f_S)\) es el conjunto que hace que \(m_1x_1 + \cdots + m_nx_n = 0\).
\(R(S)\) se denomina el \textbf{subgrupo de relaciones de \(G\)}. Definimos \(R = \{
   r_1, \cdots, r_l \}\) como el sistema generador de \(R(S)\). El par \((R,S)\)
   se denomina \textbf{presentación de \(G\) mediante generadores y relaciones}.

\importante\propo{7.13} Sea \(G\) un grupo abeliano finitamente generado por \(S = \{
x_1, \cdots, x_n \}\). Entonces se cumple que  \(S = \{ x_1, \cdots,
-x_i, \cdots, x_n\}\) también es sistema generador, y si \(\lambda _2, \cdots, \lambda _n\) son
enteros, y \(y_1 = x_1 + \lambda_2 x_2 + \cdots + \lambda _n x_n\), entonces \(S = \{
y_1, x_2, \cdots, x_n \}\) también es un sistema generador.

\importante\propo{7.14} Sea \((S,R)\) una presentación de \(G\), entonces se pueden
obtener los coeficientes de torsión y el coeficiente de Betti mediante un ejemplo (aborrezco el 
algoritmo en forma de matriz).

Tomamos el sistema de la izquierda. El término más pequeño es \(3x\). 
Procuramos obtener que todos los términos de 
\(x\) sean divisibles entre sí. En este caso lo hemos conseguido en un paso,
pero si no se consigue, se repite tantas veces como sea necesario.
$$\begin{cases}
7x+11y+13z+5u =  0 \\ 
3x+11y+7z+14u = 0 \\ 
5x+7y+11z+7u = 0 \\ 
\end{cases} \xrightarrow[r_1:r_1-2r_2]{r_3:r_3-r_1}  
\begin{cases}
x+y-z-u =  0 \\
3x+11y+7z+14u = 0 \\
2x+2y+4z+3u = 0 \\
\end{cases}$$
Ahora creamos la variable \(a = x + y - z - u\), y reescribimos el resto de
ecuaciones en función de \(a\).
$$\begin{cases}
x+y-z-u =  0 \\ 
3(x+y-z-u)+8y+10z +17u= 0 \\ 
2(x+y-z-u) + 6z +5u = 0 \\ 
\end{cases} \rightarrow 
\begin{cases}
\textcolor{blue}{a =  0} \\ 
3a+8y+10z+17u = 0 \\ 
2a + 6z +5u = 0 \\ 
\end{cases} \rightarrow
\begin{cases} 
10z + 17u + 8y= 0 \\ 
6z +5u = 0 \\ 
\end{cases}$$
Ahora tenemos un sistema ya reducido a dos relaciones y tres elementos. Aunque \(5u\)
es el menor elemento, vamos a seguir con \(6z\), ya que el resultado será el
mismo (demostración). El primer paso será obtener factores múltiplos entre sí para la columna
de \(z\).
$$\begin{cases}
10z+17u+8y=0\\
6z+5u=0\\
\end{cases} \xrightarrow{r_1:r_1-r_2}
\begin{cases}
4z+12u+8y=0\\
6z+5u=0\\
\end{cases} \xrightarrow{r_2:r_2-r_1} 
\begin{cases}
4z+12u+8y=0\\
2z-7u=0\\
\end{cases}$$
Ahora \(r_2\) ya puede eliminarse. Sin embargo, los términos \(2z\) y \(7u\)
 no son múltiplos, así que habrá que reducir la expresión.
$$\begin{cases}
4(z-4u)+28u+8y=0\\
2(z-4u)+u=0\\
\end{cases}
\rightarrow 
\begin{cases}
28u+4b+8y=0\\
u+2b=0\\
\end{cases}
 \rightarrow 
\begin{cases}
28(u+2b)-52b+8y=0\\
u+2b=0\\
\end{cases}
\rightarrow 
\begin{cases}
28c-52b+8y=0\\
\textcolor{blue}{c=0}\\
\end{cases}$$
Ahora sólo nos queda \(8y-52b=0\), que se simplifica a \(8(y-7b) + 4b=0
\rightarrow 4b+8d = 0 \rightarrow 4(b+2e)=0\). En este último caso, por ser la
última ecuación, que viene dada por dos variables, tenemos que \(\beta(G) = 1\), y obtenemos \(4f=0\), luego \(G \simeq \mathbb{Z}/ 4 \mathbb{Z} \times
\mathbb{Z}\).

\dem{$$
\begin{cases} 
10z + 17u + 8y= 0 \\ 
6z +5u = 0 \\ 
\end{cases} \rightarrow 
\begin{cases} 
5u + 6z = 0 \\
17u+10z+8y=0\\ 
\end{cases} \xrightarrow{r_2:r_2-3r_1}   
\begin{cases} 
5u + 6z = 0 \\
2u-8z+8y=0\\ 
\end{cases}
 \xrightarrow{r_1:r_1-2r_2}   
\begin{cases} 
u + 22z-16y=0 \\
2u-8z+8y=0\\ 
\end{cases}
$$
Ya tenemos el sistema preparado para reducir la variable.
$$\begin{cases} 
u + 22z-16y=0 \\
2(u+22z-16y)-52z+40y=0\\ 
\end{cases} \rightarrow 40y - 52x = 0 \rightarrow 40(y-x) - 12x = 0 \rightarrow 12(-x + 3e) + 4e = 0 \rightarrow  4(e + 3f) = 0
$$
Vemos que igualmente tenemos \( G \simeq \mathbb{Z}/ 4 \mathbb{Z} \times
\mathbb{Z} \)
}
\end{document}