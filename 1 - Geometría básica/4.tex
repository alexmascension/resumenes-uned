\defi{4.1} Sean $r,l$ dos rectas con un punto $V$ en común. Sean $\overline{r}$ y $\overline{l}$ dos semirrectas determinadas por $V$ en $r$ y $l$. El par $\{\overline{l}, \overline{r} \}$ es un \textbf{ángulo}. $V$ es el vértice del ángulo y $\overline{l}$ y $\overline{r}$ son los lados del ángulo. El ángulo se designa por $\angle \{\overline{l}, \overline{r} \}$ o, si no hay lugar a confusión, $\angle V$. Así, por ejemplo, dado un triángulo $\triangle PQR$, $\angle P$ es el ángulo formado por $P$ con $[P,Q]$ y $[P,R]$.

\obs{4.4} Si $r = l$, y $\overline{r}_1$ y $\overline{r}_2$ son las semirrectas determinadas por $V$, entonces, en estas circunstancias, el ángulo $\angle\{ \overline{r}_1, \overline{r}_2 \}$ se denomina \textbf{ángulo llano} y $\angle\{ \overline{r}_1, \overline{r}_1 \}$ se denomina \textbf{ángulo nulo}.

\defi{4.5} Un ángulo $\angle \{\overline{l}, \overline{r} \}$ y un ángulo $\angle \{\overline{l}', \overline{r}' \}$ son \textbf{congruentes} si existe una isometría $g$ tal que $g(\{\overline{l}, \overline{r} \}) = \{\overline{l}', \overline{r}'\} $. Todos los ángulos que son congruentes forman una \textbf{clase de congruencia} de ángulos. Empleando la notación de vértices, la congruencia se denota como $\angle A = \angle B$.

\obs{4.6/4.8} Si  $\angle \{\overline{l}, \overline{r} \}$ tiene vértice $V$ y  $\angle \{\overline{l}', \overline{r}' \}$ tiene vértice $V'$, y $g$ es una isometría tal que $g(  \{\overline{l}, \overline{r} \}) =   \{\overline{l}', \overline{r}' \}$, entonces $g(V) = V'$. Asimismo, si existe una isometría $h$ que hace $h(V) = V'$, entonces  $h(  \{\overline{l}, \overline{r} \}) =   \{\overline{l}', \overline{r}' \}$.

\ejem{4.9} Consideramos las rectas $a \neq b$ que cortan en $V$, con sus respectivas semirrectas 
$\overline{a}_1, \overline{a}_2, \overline{b}_1, \overline{b}_2$. Consideramos $\angle \{\overline{a}_1, \overline{b}_1 \}$ y elegimos los puntos $A \in \overline{a}_1, B \in \overline{b}_1$ a igual distancia, $d(V,A) = d(V,B)$. Existe una recta $l \perp r_{AB}$ que pasa por $V$ (\tma{2.25/2.29}, que denominamos \textbf{bisectriz}. La bisectriz $l$ cumple que $\sigma_l(A) = B, \sigma_l(\overline{a}_1) = \overline{b}_1$ y viceversa. Además, si $\overline{l}$ es la semirrecta que corta a $[A,B]$, entonces $\angle \{\overline{a}_1, \overline{l} \} = \angle \{\overline{b}_1, \overline{l} \}$.

\tma{4.11} 

	 
	 