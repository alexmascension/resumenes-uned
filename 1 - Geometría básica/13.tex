\defi{13.1} Un \textbf{poliedro} $\PP$ es un conjunto finito de polígonos $\{C_kw\}$. Los polígonos de $\PP$ se llaman caras, los lados del polígono se llaman aristas o lados, y los vértices tienen el mismo nombre. Todo poliedro cumple:
\begin{itemizex}
	\item Dos caras de un poliedro o bien no se cortan, y tienen un único vértice en como, o un lado en común.
	\item Cada arista es un lado de dos polígonos de $\PP$.
	\item Las caras que comparten un vértice en común $V$ se pueden ordenar en una sucesión $C_1, \cdots, C_r$ de modo que $C_i$ y $C_{i+1}$ son adyacentes. 
	\item Dadas dos caras $C_i, C_j$ existe una sucesión finita de caras $C_1, \cdots, C_r$ tal que $C_i = C_1, C_r = C_j$.
\end{itemizex}

\defi{13.2} Un poliedro es \textbf{convexo} si toda recta no contenida en ninguno de los planos que contienen a las caras corta a lo más en dos puntos a las caras.

\defi{13.3} Un \textbf{ciclo poligonal} $\C$ es un conjunto finito de segmentos (lados) con un conjunto finito de puntos (vértices) que verifican
\begin{itemizex}
	\item Dos segmentos o no se cortan o tienen un extremo en común.
	\item Los lados de $\C$ se pueden escribir como una sucesión finita de la forma $[V_1,V_2], [V_2,V_3],\cdots, [V_{r-1},V_r],[V_r,V_1]$.
\end{itemizex}

\defi{13.4} Sea $\L$ un conjunto formado por algunos lados de $\PP$. Dadas dos caras $P$ y $P'$ de $\PP$ decimos que están \textbf{conectadas} en $\PP - \L$ si existe una sucesión de polígonos de $\PP$, $P = P_1, \cdots, P_r = P'$ de modo que $P_i$ y $P_{i+1}$ tienen un lado en común que no está en $\L$. Si $C$ es una cara de $\PP$, la componente conexa de $\PP - \L$ que contiene a $C$ es el subconjunto de $\PP$ formado por los polígonos de $\PP$ que están conectados con $C$ en $\PP-\L$.

\tma{13.5} Sea $\PP$ un poliedro convexo y $\C$ un ciclo de $\PP$, entones hay exactamente dos componentes convexas en $\PP-\C$.

