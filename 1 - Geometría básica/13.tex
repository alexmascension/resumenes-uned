\defi{13.1} Un \textbf{poliedro} $\PP$ es un conjunto finito de polígonos $\{C_kw\}$. Los polígonos de $\PP$ se llaman caras, los lados del polígono se llaman aristas o lados, y los vértices tienen el mismo nombre. Todo poliedro cumple:
\begin{itemizex}
	\item Dos caras de un poliedro o bien no se cortan, y tienen un único vértice en como, o un lado en común.
	\item Cada arista es un lado de dos polígonos de $\PP$.
	\item Las caras que comparten un vértice en común $V$ se pueden ordenar en una sucesión $C_1, \cdots, C_r$ de modo que $C_i$ y $C_{i+1}$ son adyacentes. 
	\item Dadas dos caras $C_i, C_j$ existe una sucesión finita de caras $C_1, \cdots, C_r$ tal que $C_i = C_1, C_r = C_j$.
\end{itemizex}

\defi{13.2} Un poliedro es \textbf{convexo} si toda recta no contenida en ninguno de los planos que contienen a las caras corta a lo más en dos puntos a las caras.

\defi{13.3} Un \textbf{ciclo poligonal} $\C$ es un conjunto finito de segmentos (lados) con un conjunto finito de puntos (vértices) que verifican
\begin{itemizex}
	\item Dos segmentos o no se cortan o tienen un extremo en común.
	\item Los lados de $\C$ se pueden escribir como una sucesión finita de la forma $[V_1,V_2], [V_2,V_3],\cdots, [V_{r-1},V_r],[V_r,V_1]$.
\end{itemizex}

\defi{13.4} Sea $\L$ un conjunto formado por algunos lados de $\PP$. Dadas dos caras $P$ y $P'$ de $\PP$ decimos que están \textbf{conectadas} en $\PP - \L$ si existe una sucesión de polígonos de $\PP$, $P = P_1, \cdots, P_r = P'$ de modo que $P_i$ y $P_{i+1}$ tienen un lado en común que no está en $\L$. Si $C$ es una cara de $\PP$, la componente conexa de $\PP - \L$ que contiene a $C$ es el subconjunto de $\PP$ formado por los polígonos de $\PP$ que están conectados con $C$ en $\PP-\L$.

\tma{13.5} Sea $\PP$ un poliedro convexo y $\C$ un ciclo de $\PP$, entones hay exactamente dos componentes convexas en $\PP-\C$.

\tma{13.9 [Descartes-Euler]} Sea $\P$ un poliedro convexo, con $c$ caras, $l$ lados y $v$ vértices, entonces:
$c-l+v = 2$

\defi{13.11} Un \textbf{poliedro regular} es un poliedro convexo contodas las caras congruentes a un mismo polígono regular y cada vértice está en un mismo número de caras. Decimos que un poliedro regular tiene tipo $\{n, m\}$ si sus caras son polígonos regulares con $n$ lados y cada vértice es vértice exactamente de $m$ caras.

\begin{tabular}{ccccc}
	Nombre & Tipo & $c$ & $l$ & $v$ \\\midrule
	Tetraedro & $\{3,3\} $& 4 & 6 & 4 \\
	Octaedro & $\{3,4\}$ & 8 & 12 & 6 \\
	Cubo & $\{4,3\}$ & 6 & 12 & 8 \\
	Dodecaedro & $\{3,5\}$ & 20 & 30 & 12 \\
	Icosaedro & $\{5,3\}$ & 12 & 30 & 20 \\
\end{tabular}

\tma{13.14} Dado un real $l>0$ existe un poliedro regular de tipo $\{3,3\},\{3,4\},\{4,3\},\{3,5\},\{5,3\}$, cuya arista mide $l$. Además, si $\PP_1$ y $\PP_2$ son dos poliedros del mismo tipo y con la misma longitud de arista entonces existe una isometría $\eta$ tal que $\eta(\PP_1) = \PP_2$.

\tma{13.16} Sea $V, W$ dos vértices de un poliedro regular $\PP, a, b$ dos aristas, de modo que $a$ tiene por uno de sus extremos $V$ y $b$ tiene por extremo $W$, por último sea $C_1$ una cara que tiene a $a$ como uno de sus lados y $C_2$ una cara que tiene a $b$ como lado. Existe una simetría $\theta$ de $\PP$ tal que
\[\theta(V) = W \;\; \theta(a)=b \;\; \theta(C_1) = C_2  \]

\defi{} Dado un polígono $\PP$ regular, el polígono \textbf{dual} de $\PP$, o $\PP*$, es aquel formado por la unión de los centros de las caras que forman los triedos (tres polígonos que comparten el mismo vértice $V$).

\defi{} Si consideramos el plano $\pi$ ortogonal a $r_{VW}$, siendo $V,W$ los dos vértices de un lado en común entre dos caras$P_1, P_2$ de $\PP$. $\pi$ pasa, por ejemplo, por $\m[V,W]$. Llamamos \textbf{ángulo diédrico} al ángulo de $\pi$ con vértice en $\m[V,W]$ y cuyos lados contienen a los segmentos que son las intersecciones de $P_1$ y $P_2$ con $\pi$.

\subsection{Simetrías de poliedros}
Existen tres tipos de simetrías

\textbf{Rotaciones}
\begin{itemizex}
	\item Rotaciones con eje ortogonal a una cara $C$ de $\PP$ y pasa por el el centro de $C$; con ángulos de rotación $2\pi r/n\;,r = 1, \cdots, n-1$.
	\item Rotaciones cuyo eje pasa por un vértice $V$ de $\PP$ y es ortogonal al polígono formado por los centros de las caras de $\PP$ que tienen a $V$ como uno de sus vértices; con ángulos de rotación $2\pi r/m\;,r = 1, \cdots, m-1$.
	\item Medias vueltas con eje $e$ que pasa por el punto medio $M$ de una arista $a$ de $\PP$. Además $e$ es ortogonal a $a$, así deja invariante la arista $a$ aunque intercambia sus extremos. $e$ es la bisectriz del ángulo formado por las dos caras $C_1, C_2$ que comparten $a$ y que pasa por $M$; permutando así $C_1$ y $C_2$.
\end{itemizex}

\textbf{Reflexiones}
\begin{itemizex}
	\item Sea $C$ una cara de $\PP$, que está contenida en un plano $\pi$.
	Cada reflexión $\sigma$ en $\pi$ se extiende a una reflexión $\eta$ en el espacio. El plano de reflexión para $\eta$ es el plano ortogonal a $\pi$ que lo corta en el eje de reflexión. Cada poliedro tiene, por cara, $n$ reflexiones si la cara es pentágono o triángulo y 2 si es cuadrado.
	\item Para el \textit{octaedro} existe una reflexión sobre el cuadrado interno que forman las dos pirámides.
\end{itemizex}

\textbf{Reflexiones-rotaciones}
\begin{itemizex}
	\item Tipo A: tanto la reflexión como la rotación son simetrías. 
		\subitem Cubo y octaedro: rotación con ángulo $\pi/2$ o $\pi$ y con ejes que pasa por el centro de las caras opuestas (cubo) o por los puntos opuestos de las pirámides, o del cuadrado interno (octaedro).
		\subitem Dodecaedro e icosaedro: rotación con ángulos $\pi/3, \pi/5, 3\pi/5$ y con eje que pasa por los puntos medios de dos caras opuestas o dos vértices opuestos.
	\item Tipo B: ni la reflexión ni la rotación son simetrías, pero la combinación de ambas sí.
		\subitem Tetraedro: rotación con ángulo $\pi/2$ y con eje que pasa por los puntos medios de las aristas.
		\subitem Cubo: rotación con ángulo $\pi/3$ y con eje que pasa por los puntos opuestos del poliedro.
		\subitem Octaedro: rotación con ángulo $\pi/3$ y con eje que pasa por los puntos medios de dos caras opuestas.
		\subitem Dodecaedro e icosaedro: rotación con ángulo $\pi$ y con eje que pasa por los puntos medios de dos caras opuestas o dos vértices opuestos.
\end{itemizex}